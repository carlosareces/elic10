\documentclass[compress,color=usenames]{beamer}

\newcommand{\mytitlenbr}{1}
\newcommand{\mytitle}{Image Archive}

%\documentclass[compress,color=usenames,handout]{beamer}

%\usepackage{pgfpages}
%\pgfpagelayout{4 on 2}[a4paper,border shrink=5mm]

\usepackage{graphicx}
\usepackage{amsfonts,amssymb}
\usepackage{latexsym}
\usepackage{mdwtab}
\usepackage{xspace}
\usepackage{tikz}
\usetikzlibrary{shapes,snakes}
\usetikzlibrary{petri}

\DefineNamedColor{named}{Periwinkle}{cmyk}{0.57,0.55,0,0}
\DefineNamedColor{named}{Plum}{cmyk}{0.50,1,0,0}
\DefineNamedColor{named}{Red}{cmyk}{0,1,1,0}

\newcommand{\mH}[1]{\textcolor{Plum}{#1}}
\newcommand{\mT}[1]{\textcolor{Periwinkle}{#1}}

\newcommand{\tup}[1]{\langle #1 \rangle}

\newcommand{\dd}{{:}}
\newcommand{\I}{\mathcal{I}}
\newcommand{\csetsc}[2]{\{#1 \mid #2\}}
\newcommand{\cset}[1]{\{#1\}}

\newcommand{\CON}{\textsf{CON}\xspace}
\newcommand{\ROL}{\textsf{ROL}\xspace}
\newcommand{\IND}{\textsf{IND}\xspace}
\newcommand{\PROP}{\textsf{PROP}\xspace}
\newcommand{\lang}{\mathcal{L}\xspace}

\newcommand{\mytt}[1]{\textsf{\scriptsize{#1}}}
\newcommand{\mytts}[1]{\textsf{\scriptsize{#1}}}

%\usefonttheme{serif}

\mode<presentation>
 {
 \usetheme{lined}
 }

\setbeamertemplate{navigation symbols}{}

\newcommand{\F}{\mathop{\mathsf{F}\vphantom{a}}\nolimits}
\newcommand{\G}{\mathop{\mathsf{G}\vphantom{a}}\nolimits}
\newcommand{\X}{\mathop{\mathsf{X}\vphantom{a}}\nolimits}

\newcommand{\Blue}[1]{\textcolor{blue}{#1}}
\newcommand{\Red}[1]{\textcolor{red}{#1}}
\newcommand{\Green}[1]{\textcolor{PineGreen}{#1}}

\title[GLN y Aplicaciones]{\Huge Generaci\'on de Lenguaje Natural y Aplicaciones}
%\mH{Lecture \#\mytitlenbr:} \mytitle}

\author[Areces \& Benotti]{
 Carlos Areces y Luciana Benotti\\[1ex]
\normalsize \url{{carlos.areces, luciana.benotti}@gmail.com}}

\institute[INRIA / UNC]{
INRIA Nancy Grand Est, Nancy, France\\
Universidad Nacional de C\'ordoba, C\'ordoba, Argentina}

\date{ELiC 2010 - Buenos Aires - Argentina}

\begin{document}

\beamerdefaultoverlayspecification{}

\begin{frame}[plain]
 \titlepage
\end{frame}

\begin{frame}
\frametitle{Generando Expresiones Referenciales}

\begin{quote}

\textbf{He claims record}\medskip

The 22-year-old computer science undergraduate from Bath is
claiming a world record for the longest distance ridden on a
unicycle in 24 hours.\medskip

A unicycling student covered exactly 282 miles at Aberystwy
University's athletics track.\medskip

Sam Wakeling was aiming to beat the existing record of 235.3
miles.
\end{quote}

\end{frame}

\begin{frame}
\frametitle{Generando Expresiones Referenciales}

\begin{quote}

\textbf{\mH{He} claims record}\medskip

\mH{The 22-year-old computer science undergraduate from Bath} is
claiming a world record for the longest distance ridden on a
unicycle in 24 hours.\medskip

\mH{A unicycling student} covered exactly 282 miles at Aberystwy
University's athletics track.\medskip

\mH{Sam Wakeling} was aiming to beat the existing record of 235.3
miles.
\end{quote}

\end{frame}

\begin{frame}
\frametitle{Generando Expresiones Referenciales}

\begin{quote}
\textbf{\mH{Unicycling student} claims record}\medskip

\mH{A student} is claiming a world record for the longest distance ridden
on a unicycle in 24 hours.\medskip

\mH{Sam Wakeling} covered exactly 282 miles at Aberystwyth
University's athletics track.\medskip

\mH{The 22-year-old computer science undergraduate from Bath} was
aiming to beat the existing record of 235.3 miles.
\end{quote}

\end{frame}

\begin{frame}
\frametitle{Terminolog\'ia}

\begin{itemize}
\item \mH{Referring Expression}
A linguistic expression (typically an NP) that a speaker uses to
identify a (discourse) entity to the hearer.

\item \mH{Referent}
The entity to which the speaker refers by a referring expression.

\item \mH{Reference}
The process of identifying an entity by using a referring expression.

\end{itemize}
\end{frame}


\begin{frame}
\frametitle{Reference and Discourse}

A given entity can be referred to in various ways:

\begin{enumerate}
\item 
Noam Chomsky has given a talk today.

\item
One of the people working at MIT has given a talk today.

\item 
A person who is working at MIT has given a talk today.

\item 
A person has given a talk today.
\end{enumerate}
\end{frame}

\begin{frame}
\frametitle{Reference and Discourse}

Reference and linguistic form

The linguistic form of a referring expression reflects the current
state of the discourse (and the speaker's beliefs about the hearer's
discourse model)\pause

Typically:

new discourse referents are introduced by indefinite NPs ({``}a
cat'')

old discourse referents are referred to by definite NPs and
pronouns ({``}the cat''/''it'')

\end{frame}

\begin{frame}
\frametitle{Prince (1981, 1992): Linguistic Form \& Familiarity Scale}

Assumed Familiarity:

From the point of view of the speaker/writer:

Which assumptions about the hearer/reader influence the choice of
the referring expression?

From the point of view of the hearer/reader:

Which conclusion is he/she going to draw from the choice of the
referring expression?

\end{frame}

\begin{frame}
\frametitle{Dimensions of Familiarity (Prince 1981, 1992)}

status of the referent

(assumed by the speaker)

hearer-new

hearer-old

discourse-new

discourse-old

\end{frame}

\begin{frame}
\frametitle{Dimensions of Familiarity (Prince 1981, 1992)}

status of the referent

(assumed by the speaker)

hearer-new

hearer-old

discourse-new

discourse-old

brand-new

brand-new: introduction of a new discourse referent representing

an unknown entity (a student)

\end{frame}

\begin{frame}
\frametitle{Dimensions of Familiarity (Prince 1981, 1992)}

status of the referent

(assumed by the speaker)

hearer-new

hearer-old

discourse-new

discourse-old

brand-new

---

brand-new: introduction of a new discourse referent representing

an unknown entity (a student)

\end{frame}

\begin{frame}
\frametitle{Dimensions of Familiarity (Prince 1981, 1992)}

status of the referent

(assumed by the speaker)

hearer-new

hearer-old

discourse-new

discourse-old

brand-new

unused

---

brand-new: introduction of a new discourse referent representing

an unknown entity (a student)

unused: introduction of a new discourse referent representing a

known entity (Queen Elisabeth)

\end{frame}

\begin{frame}
\frametitle{Dimensions of Familiarity (Prince 1981, 1992)}

status of the referent

(assumed by the speaker)

hearer-new

hearer-old

discourse-new

discourse-old

brand-new

unused

---

evoked

brand-new: introduction of a new discourse referent representing

an unknown entity (a student)

unused: introduction of a new discourse referent representing a

known entity (Queen Elisabeth)

evoked: an entity is related to one which

has been referred to before (in the discourse)

The 22-year old computer science undergraduate from Bath

or is present in the situation

(you)

\end{frame}

\begin{frame}
\frametitle{Dimensions of Familiarity (Prince 1981, 1992)}

status of the referent

(assumed by the speaker)

hearer-new

hearer-old

discourse-new

discourse-old

brand-new

unused

---

evoked

brand-new: introduction of a new discourse referent representing

an unknown entity (a student)

unused: introduction of a new discourse referent representing a

known entity (Queen Elisabeth)

evoked: an entity is related to one which

has been referred to before (in the discourse)

The 22-year old computer science undergraduate from Bath

or is present in the situation

(you)

inferrable: introduction of a new discourse referent whose relation

to a known entity is inferrable (like hearer-old but neither

discourse-new nor discourse-old)

(Peter walked towards the house. The door was open.)

Caroline Sporleder

\end{frame}

\begin{frame}
\frametitle{Prince's Familiarity Hierarchy}

Assumed Familiarity

New

Brand$-$new

Inferrable

Unused

U

(Noncontaining)

Inferrable

I

Containing

Inferrable

C

I

Evoked

(Textually) Situationally

Evoked

Evoked

ES

E

Brand$-$new

Brand$-$new

(unanchored) Anchored

BN A

BN

\end{frame}

\begin{frame}
\frametitle{Prince's Familiarity Hierarchy}

Assumed Familiarity

New

Brand$-$new

Inferrable

Unused

U

(Noncontaining)

Inferrable

I

Containing

Inferrable

C

I

Evoked

(Textually) Situationally

Evoked

Evoked

ES

E

Brand$-$new

Brand$-$new

(unanchored) Anchored

BN

BN A

Yesterday I got on a bus.

\end{frame}

\begin{frame}
\frametitle{Prince's Familiarity Hierarchy}

Assumed Familiarity

New

Brand$-$new

Inferrable

Unused

U

(Noncontaining)

Inferrable

I

Containing

Inferrable

C

I

Evoked

(Textually) Situationally

Evoked

Evoked

ES

E

Brand$-$new

Brand$-$new

(unanchored) Anchored

BN A

BN

Somebody who works with Peter says he knows your sister.

\end{frame}

\begin{frame}
\frametitle{Prince's Familiarity Hierarchy}

Assumed Familiarity

New

Brand$-$new

Inferrable

Unused

U

(Noncontaining)

Inferrable

I

Containing

Inferrable

C

I

Evoked

(Textually) Situationally

Evoked

Evoked

S

E

E

Brand$-$new

Brand$-$new

(unanchored) Anchored

BN

BN A

Noam Chomsky went to Penn.

\end{frame}

\begin{frame}
\frametitle{Prince's Familiarity Hierarchy}

Assumed Familiarity

New

Brand$-$new

Inferrable

Unused

U

(Noncontaining)

Inferrable

I

Containing

Inferrable

C

I

Evoked

(Textually) Situationally

Evoked

Evoked

S

E

E

Brand$-$new

Brand$-$new

(unanchored) Anchored

BN A

BN

Somebody who works with Peter says he knows your sister.

\end{frame}

\begin{frame}
\frametitle{Prince's Familiarity Hierarchy}

Assumed Familiarity

New

Brand$-$new

Inferrable

Unused

U

(Noncontaining)

Inferrable

I

Containing

Inferrable

C

I

Evoked

(Textually) Situationally

Evoked

Evoked

ES

E

Brand$-$new

Brand$-$new

(unanchored) Anchored

BN A

BN

Excuse me do you have the time?

\end{frame}

\begin{frame}
\frametitle{Prince's Familiarity Hierarchy}

Assumed Familiarity

New

Brand$-$new

Inferrable

Unused

U

(Noncontaining)

Inferrable

I

Containing

Inferrable

C

I

Evoked

(Textually) Situationally

Evoked

Evoked

S

E

E

Brand$-$new

Brand$-$new

(unanchored) Anchored

BN A

BN

Yesterday I got on a bus. The driver was drunk.

\end{frame}

\begin{frame}
\frametitle{Prince's Familiarity Hierarchy}

Assumed Familiarity

New

Brand$-$new

Inferrable

Unused

U

(Noncontaining)

Inferrable

I

Containing

Inferrable

C

I

Evoked

(Textually) Situationally

Evoked

Evoked

ES

E

Brand$-$new

Brand$-$new

(unanchored) Anchored

BN

BN A

The pages of the book which I just bought, were falling out.

\end{frame}

\begin{frame}
\frametitle{Prince's Familiarity Hierarchy}

Assumed Familiarity

New

Brand$-$new

Inferrable

Unused

U

(Noncontaining)

Inferrable

I

Containing

Inferrable

C

I

Evoked

(Textually) Situationally

Evoked

Evoked

ES

E

Brand$-$new

Brand$-$new

(unanchored) Anchored

BN A

BN

E

ES

$>$ U $>$ I $>$ I C $>$ BN A $>$ BN

\end{frame}

\begin{frame}
\frametitle{Prince's Familiarityscale}

E

ES

$>$ U $>$ I $>$ I C $>$ BN A $>$ BN

1

Noam Chomsky has given a talk today. (U)

2

One of the people working at MIT has given a talk today.

(I C )

3

A person who is working at MIT has given a talk today.

(BN A )

4

A person has given a talk today. (BN)

\end{frame}

\begin{frame}
\frametitle{Generating Referring Expressions: Rule-based}

Approaches

\end{frame}

\begin{frame}
\frametitle{Pick your Referring Expression}

\end{frame}

\begin{frame}
\frametitle{Generating Referring Expressions}

Important for . . .

generation (concept-to-text, text-to-text)

Need to distinguish

identification of target entity to hearer (discriminate between

target entity and other entities in discourse)

other communicative goals

I met an old friend yesterday. vs. The scoundrel is the one

who betrayed us.

\end{frame}

\begin{frame}
\frametitle{Concept-to-text Generation}

ID

e1

e2

e3

e4

what

coin

coin

helmet

helmet

material

gold

silver

silver

bronze

origin

Roman

Greek

Roman

Etruscan

where found

Pompeii

Chios

Rome

Pisa

date

50 BC

600 BC

200 AD

400 BC

\end{frame}

\begin{frame}
\frametitle{Concept-to-text Generation}

ID

e1

e2

e3

e4

what

coin

coin

helmet

helmet

material

gold

silver

silver

bronze

origin

Roman

Greek

Roman

Etruscan

where found

Pompeii

Chios

Rome

Pisa

date

50 BC

600 BC

200 AD

400 BC

The silver coin is Greek. It was found on Chios and {\O} dates from

around 600 BC. The other coin is Roman. It is gold and {\O} dates

from around 50 BC. It was found in Pompeii. . .

\end{frame}

\begin{frame}
\frametitle{Concept-to-text Generation}

ID

e1

e2

e3

e4

what

coin

coin

helmet

helmet

material

gold

silver

silver

bronze

origin

Roman

Greek

Roman

Etruscan

where found

Pompeii

Chios

Rome

Pisa

date

50 BC

600 BC

200 AD

400 BC

The silver coin is Greek. It was found on Chios and {\O} dates from

around 600 BC. The other coin is Roman. It is gold and {\O} dates

from around 50 BC. It was found in Pompeii. . .

Here are two coins: the silver one is Greek . . .

Caroline Sporleder

\end{frame}

\begin{frame}
\frametitle{Rule-base Generation of Referring Expressions (REs)}

(cf. Dale (1989), Reiter \& Dale (1992), Dale \& Reiter (1995))

EPICURE system

generate cookery recipes

deep generation

three levels of semantic representation:

flexible knowledge base (KB) representing physical objects in a

specific state

deep semantic structure (DS) of a RE (recoverable by hearer)

surface semantic structure (SS) of an RE (to be realised by the

generation grammar)

to construct an RE:

1

2

construct deep semantic structure for a KB entity

i.e. what do you want to say about the entity?

from the DS construct a surface semantic structure

i.e. how dow you want to say it (pronominalisation etc.)

\end{frame}

\begin{frame}
\frametitle{Rule-base Generation of Referring Expressions (REs)}

(cf. Dale (1989), Reiter \& Dale (1992), Dale \& Reiter (1995))

EPICURE system

generate cookery recipes

deep generation

three levels of semantic representation:

flexible knowledge base (KB) representing physical objects in a

specific state

$\Rightarrow$ in this domain, discourse entities change!

deep semantic structure (DS) of a RE (recoverable by hearer)

surface semantic structure (SS) of an RE (to be realised by the

generation grammar)

to construct an RE:

1

2

construct deep semantic structure for a KB entity

i.e. what do you want to say about the entity?

from the DS construct a surface semantic structure

i.e. how dow you want to say it (pronominalisation etc.)

\end{frame}

\begin{frame}
\frametitle{Rule-base Generation of Referring Expressions (REs)}

\end{frame}

\begin{frame}
\frametitle{Rule-base Generation of Referring Expressions (REs)}

\end{frame}

\begin{frame}
\frametitle{Rule-base Generation of Referring Expressions (REs)}

\end{frame}

\begin{frame}
\frametitle{Rule-base Generation of Referring Expressions (REs)}

\end{frame}

\begin{frame}
\frametitle{Generation of Pronouns and Elided NPs}

Pronominalisation:

basically Centering-based approach (Cb s can be pronominalised)

Elided (empty) NPs

Cb s can be elided if they fill an optional grammatical role (e.g.

indirect object)

Fry the onionsi .

Add the garlic {\O}i .

\end{frame}

\begin{frame}
\frametitle{Generation of Definite NPs}

take into account Gricean Maxims:

\end{frame}

\begin{frame}
\frametitle{Generation of Definite NPs}

take into account Gricean Maxims:

Quality

Say the truth.

Quantity

Be as informative as possible.

Relevance

Be relevant.

Manner

Be brief. Avoid ambiguity. Be orderly.

\end{frame}

\begin{frame}
\frametitle{Generation of Definite NPs}

take into account Gricean Maxims:

Quality

Say the truth.

Quantity

Be as informative as possible.

Relevance

Be relevant.

Manner

Be brief. Avoid ambiguity. Be orderly.

\end{frame}

\begin{frame}
\frametitle{Generation of Definite NPs}

take into account Gricean Maxims:

Quality

Say the truth.

Quantity

Be as informative as possible.

Relevance

Be relevant.

Manner

Be brief. Avoid ambiguity. Be orderly.

$\Rightarrow$ Choose the shortest RE that discriminates the target entity

from all other entities in the discourse model (the distractors).

\end{frame}

\begin{frame}
\frametitle{Generation of Definite NPs}

Strategy

entities in the KB are described by attribute-value pairs (AVPs)

$\Rightarrow$ realise the smallest set of values that singles out the entity

\end{frame}

\begin{frame}
\frametitle{Generation of Definite NPs}

Strategy

entities in the KB are described by attribute-value pairs (AVPs)

$\Rightarrow$ realise the smallest set of values that singles out the entity

ID

e1

e2

e3

e4

e5

what

coin

coin

coin

helmet

helmet

material

gold

silver

silver

silver

bronze

origin

Roman

Roman

Greek

Roman

Etruscan

where found

Pompeii

Rome

Chios

Rome

Pisa

date

50 BC

30 AD

600 BC

200 AD

400 BC

\end{frame}

\begin{frame}
\frametitle{Generation of Definite NPs}

Strategy

entities in the KB are described by attribute-value pairs (AVPs)

$\Rightarrow$ realise the smallest set of values that singles out the entity

ID

e1

e2

e3

e4

e5

what

coin

coin

coin

helmet

helmet

material

gold

silver

silver

silver

bronze

origin

Roman

Roman

Greek

Roman

Etruscan

where found

Pompeii

Rome

Chios

Rome

Pisa

date

50 BC

30 AD

600 BC

200 AD

400 BC

\end{frame}

\begin{frame}
\frametitle{Generation of Definite NPs}

Strategy

entities in the KB are described by attribute-value pairs (AVPs)

$\Rightarrow$ realise the smallest set of values that singles out the entity

ID

e1

e2

e3

e4

e5

what

coin

coin

coin

helmet

helmet

material

gold

silver

silver

silver

bronze

origin

Roman

Roman

Greek

Roman

Etruscan

where found

Pompeii

Rome

Chios

Rome

Pisa

date

50 BC

30 AD

600 BC

200 AD

400 BC

Good:

the gold coin (the golden one, the object made of gold)

the coin found in Pompeii

the coin dating from 50 BC

\end{frame}

\begin{frame}
\frametitle{Generation of Definite NPs}

Strategy

entities in the KB are described by attribute-value pairs (AVPs)

$\Rightarrow$ realise the smallest set of values that singles out the entity

ID

e1

e2

e3

e4

e5

what

coin

coin

coin

helmet

helmet

material

gold

silver

silver

silver

bronze

origin

Roman

Roman

Greek

Roman

Etruscan

where found

Pompeii

Rome

Chios

Rome

Pisa

date

50 BC

30 AD

600 BC

200 AD

400 BC

Good:

the gold coin (the golden one, the object made of gold)

the coin found in Pompeii

the coin dating from 50 BC

Not so good:

the gold coin found in Pompeii (too informative)

the Roman coin (not unique)

Caroline Sporleder

\end{frame}

\begin{frame}
\frametitle{Generation of Definite NPs}

Search strategy:

compute the discriminatory power of each AVP and choose the one

with the highest until the entity is singled out

Distractors:

U = x1 , x2 , ..., xn

Discriminatory power of an AVP given U:

F ($<$ a, v $>$, U) = n$-$k

1$\leq$k $\leq$n

n$-$1

where k is that number of distractors for which $<$ a, v $>$ is also

true.

\end{frame}

\begin{frame}
\frametitle{Generation of Definite NPs}

Example

ID

e1

e2

e3

e4

e5

what

coin

coin

coin

helmet

helmet

material

gold

silver

silver

silver

bronze

origin

Roman

Roman

Greek

Roman

Etruscan

where found

Pompeii

Rome

Chios

Rome

Pisa

date

50 BC

30 AD

600 BC

200 AD

400 BC

\end{frame}

\begin{frame}
\frametitle{Generation of Definite NPs}

Example

ID

e1

e2

e3

e4

e5

what

coin

coin

coin

helmet

helmet

material

gold

silver

silver

silver

bronze

origin

Roman

Roman

Greek

Roman

Etruscan

F ($<$ material, gold $>$, U) =

5$-$1

5$-$1

where found

Pompeii

Rome

Chios

Rome

Pisa

date

50 BC

30 AD

600 BC

200 AD

400 BC

=1

\end{frame}

\begin{frame}
\frametitle{Generation of Definite NPs}

Example

ID

e1

e2

e3

e4

e5

what

coin

coin

coin

helmet

helmet

material

gold

silver

silver

silver

bronze

origin

Roman

Roman

Greek

Roman

Etruscan

F ($<$ material, gold $>$, U) =

F ($<$ origin, Roman $>$, U) =

5$-$1

5$-$1

5$-$2

5$-$1

where found

Pompeii

Rome

Chios

Rome

Pisa

date

50 BC

30 AD

600 BC

200 AD

400 BC

=1

=

3

4

= .75

\end{frame}

\begin{frame}
\frametitle{Generation of Definite NPs}

Example

ID

e1

e2

e3

e4

e5

what

coin

coin

coin

helmet

helmet

material

gold

silver

silver

silver

bronze

origin

Roman

Roman

Greek

Roman

Etruscan

F ($<$ material, gold $>$, U) =

F ($<$ origin, Roman $>$, U) =

5$-$1

5$-$1

5$-$2

5$-$1

where found

Pompeii

Rome

Chios

Rome

Pisa

date

50 BC

30 AD

600 BC

200 AD

400 BC

=1

=

3

4

= .75

How to choose between equally scoring AVP sets?

\end{frame}

\begin{frame}
\frametitle{Generation of Definite NPs}

Example

ID

e1

e2

e3

e4

e5

what

coin

coin

coin

helmet

helmet

material

gold

silver

silver

silver

bronze

origin

Roman

Roman

Greek

Roman

Etruscan

F ($<$ material, gold $>$, U) =

F ($<$ origin, Roman $>$, U) =

5$-$1

5$-$1

5$-$2

5$-$1

where found

Pompeii

Rome

Chios

Rome

Pisa

date

50 BC

30 AD

600 BC

200 AD

400 BC

=1

=

3

4

= .75

How to choose between equally scoring AVP sets?

$\Rightarrow$ assume attributes are ordered a priory, e.g.:

material $>$ origin $>$ where found etc.

Caroline Sporleder

\end{frame}

\begin{frame}
\frametitle{Open Issues}

\end{frame}

\begin{frame}
\frametitle{Open Issues}

$\Rightarrow$ the small dog

\end{frame}

\begin{frame}
\frametitle{Open Issues}

$\Rightarrow$ the small dog

$\Rightarrow$ the Chihuahua

\end{frame}

\begin{frame}
\frametitle{Open Issues}

$\Rightarrow$ the small dog

$\Rightarrow$ the Chihuahua

$\Rightarrow$ the 15 cm tall dog

\end{frame}

\begin{frame}
\frametitle{Open Issues}

Humans do not always choose the shortest RE . . .

$\Rightarrow$ the white bird

\end{frame}

\begin{frame}
\frametitle{Bibliography} 

\begin{itemize}\small

\item Robert Dale.
Cooking up referring expressions.
In Proceedings of ACL, 1989.

\item Robert Dale and Ehud Reiter.
Computational interpretations of the gricean maxims in the generation of referring expressions.
Cognitive Science, 19(2):233--263, 1995.

\item Ellen Prince.
Toward a taxonomy of given-new information.
In P. Cole, editor, Radical Pragmatics, pages 223--56. Academic Press, New York, 1981.

\item Ellen Prince.
The ZPG letter: subjects, definiteness, and information-status.
In S. Thompson and W. Mann, editors, Discourse descriptions: diverse analyses of a fund raising text, pages
295--325. John Benjamins, Philadelphia/Amsterdam, 1992.

\item Ehud Reiter and Robert Dale.
A fast algorithm for the generation of referring expressions.
In Proceedings of Coling, 1992.
\end{itemize}

\end{frame}

\end{document}

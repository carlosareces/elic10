\documentclass[compress,color=usenames]{beamer}

\newcommand{\mytitlenbr}{1}
\newcommand{\mytitle}{Image Archive}

%\documentclass[compress,color=usenames,handout]{beamer}

%\usepackage{pgfpages}
%\pgfpagelayout{4 on 2}[a4paper,border shrink=5mm]

\usepackage{graphicx}
\usepackage{amsfonts,amssymb}
\usepackage{latexsym}
\usepackage{mdwtab}
\usepackage{xspace}
\usepackage{tikz}
\usetikzlibrary{shapes,snakes}
\usetikzlibrary{petri}

\DefineNamedColor{named}{Periwinkle}{cmyk}{0.57,0.55,0,0}
\DefineNamedColor{named}{Plum}{cmyk}{0.50,1,0,0}
\DefineNamedColor{named}{Red}{cmyk}{0,1,1,0}

\newcommand{\mH}[1]{\textcolor{Plum}{#1}}
\newcommand{\mT}[1]{\textcolor{Periwinkle}{#1}}

\newcommand{\tup}[1]{\langle #1 \rangle}

\newcommand{\dd}{{:}}
\newcommand{\I}{\mathcal{I}}
\newcommand{\csetsc}[2]{\{#1 \mid #2\}}
\newcommand{\cset}[1]{\{#1\}}

\newcommand{\CON}{\textsf{CON}\xspace}
\newcommand{\ROL}{\textsf{ROL}\xspace}
\newcommand{\IND}{\textsf{IND}\xspace}
\newcommand{\PROP}{\textsf{PROP}\xspace}
\newcommand{\lang}{\mathcal{L}\xspace}


\newcommand{\mytt}[1]{\textsf{\scriptsize{#1}}}
\newcommand{\mytts}[1]{\textsf{\scriptsize{#1}}}

%\usefonttheme{serif}

\mode<presentation>
 {
 \usetheme{lined}
 }

\setbeamertemplate{navigation symbols}{}


\newcommand{\F}{\mathop{\mathsf{F}\vphantom{a}}\nolimits}
\newcommand{\G}{\mathop{\mathsf{G}\vphantom{a}}\nolimits}
\newcommand{\X}{\mathop{\mathsf{X}\vphantom{a}}\nolimits}

\newcommand{\Blue}[1]{\textcolor{blue}{#1}}
\newcommand{\Red}[1]{\textcolor{red}{#1}}
\newcommand{\Green}[1]{\textcolor{PineGreen}{#1}}

\begin{document}

\begin{frame}
\frametitle{Evaluaci\'on: Opcion 1}

\begin{itemize}
\item \mH{Examen Takehome}:  
\begin{itemize}
\item Con \mH{preguntas te\'oricas y ejercicios pr\'acticos} sobre los contenidos del curso.
\item Se resuelve en forma \mH{individual}. 

\item Va a aparecer en la p\'agina del curso el \mH{Lunes 2 de Agosto}. 
\medskip

\centerline{http://www.glyc.dc.uba.ar/elic2010/curso2.php}
\medskip

\item Se entrega por mail el \mH{Lunes 16 de Agosto} a \url{carlos.areces@gmail.com}
\item Envien \mH{pdf} de latex, doc, scan, etc. 
\end{itemize}
\end{itemize}
\end{frame}

\begin{frame}
\frametitle{Evaluaci\'on: Opcion 2}


\begin{itemize}

\item \mH{Projectos de Desarrollo}: 
\begin{itemize}

% planning: eliminar navegacion (=> cambiar monitoreo) 
%           agregar landmarks a la navegacion
% re: derecha / izquierda
%     mas cerca / mas lejos
%     this one  / not this one

\item Extender un aspecto (entre 5 opciones que nosotros les damos) de un generador 
de instrucciones GIVE 
   \begin{itemize}
      \item instalar el framework
      \item modificar c\'odigo Java
      \item testear en 3 mundos diferentes (provistos por nosotros)
      \item Escribir un informe de 8 p\'aginas (al menos) describiendo 
      lo que hicieron y c\'omo.  Incluyendo ejemplos de casos interesantes 
      en los mundos (pueden incluir screenshots). 
   \end{itemize}

\item Se resuelve en grupos de \mH{hasta 2 personas}. 

\item Tienen que \mH{inscribirse} hoy. 

\end{itemize}
\end{itemize}
\end{frame}

\begin{frame}
\frametitle{Evaluaci\'on: Opcion 2}


\begin{itemize}
\item \mH{Projectos de Desarrollo}: 
\begin{itemize}
\item La descripci\'on detallada va a aparecer en la p\'agina del curso el \mH{Lunes 2 de Agosto}. 
\medskip

\centerline{http://www.glyc.dc.uba.ar/elic2010/curso2.php}
\medskip

\item Se entrega por mail el \mH{Lunes 23 de Agosto} a \url{carlos.areces@gmail.com}

\item Envien \mH{pdf} del informe (de latex, doc, scan, etc), m\'as \mH{c\'odigo modificado} e 
instrucciones de cualquier cosa que debamos modificar en el framework.\pause  

\item Para asegurarnos que el proyecto est\'a en marcha les ofrecemos 2 `check points'
\begin{itemize}
  \item Viernes 9: Problemas de Instalaci\'on
  \item Viernes 16: Problemas de Implementaci\'on
\end{itemize}

\mH{Compromiso de Calidad:} Preguntas sobre esos temas (por email) ser\'an contestados en el fin 
de semana siguiente. 

\end{itemize}
\end{itemize}
\end{frame}

\end{document}

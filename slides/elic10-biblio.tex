\documentclass[compress,color=usenames]{beamer}

\newcommand{\mytitlenbr}{1}
\newcommand{\mytitle}{Image Archive}

%\documentclass[compress,color=usenames,handout]{beamer}

%\usepackage{pgfpages}
%\pgfpagelayout{4 on 2}[a4paper,border shrink=5mm]

\usepackage{graphicx}
\usepackage{amsfonts,amssymb}
\usepackage{latexsym}
\usepackage{mdwtab}
\usepackage{xspace}
\usepackage{tikz}
\usetikzlibrary{shapes,snakes}
\usetikzlibrary{petri}

\DefineNamedColor{named}{Periwinkle}{cmyk}{0.57,0.55,0,0}
\DefineNamedColor{named}{Plum}{cmyk}{0.50,1,0,0}
\DefineNamedColor{named}{Red}{cmyk}{0,1,1,0}

\newcommand{\mH}[1]{\textcolor{Plum}{#1}}
\newcommand{\mT}[1]{\textcolor{Periwinkle}{#1}}

\newcommand{\tup}[1]{\langle #1 \rangle}

\newcommand{\dd}{{:}}
\newcommand{\I}{\mathcal{I}}
\newcommand{\csetsc}[2]{\{#1 \mid #2\}}
\newcommand{\cset}[1]{\{#1\}}

\newcommand{\CON}{\textsf{CON}\xspace}
\newcommand{\ROL}{\textsf{ROL}\xspace}
\newcommand{\IND}{\textsf{IND}\xspace}
\newcommand{\PROP}{\textsf{PROP}\xspace}
\newcommand{\lang}{\mathcal{L}\xspace}


\newcommand{\mytt}[1]{\textsf{\scriptsize{#1}}}
\newcommand{\mytts}[1]{\textsf{\scriptsize{#1}}}

%\usefonttheme{serif}

\mode<presentation>
 {
 \usetheme{lined}
 }

\setbeamertemplate{navigation symbols}{}


\newcommand{\F}{\mathop{\mathsf{F}\vphantom{a}}\nolimits}
\newcommand{\G}{\mathop{\mathsf{G}\vphantom{a}}\nolimits}
\newcommand{\X}{\mathop{\mathsf{X}\vphantom{a}}\nolimits}

\newcommand{\Blue}[1]{\textcolor{blue}{#1}}
\newcommand{\Red}[1]{\textcolor{red}{#1}}
\newcommand{\Green}[1]{\textcolor{PineGreen}{#1}}

\title[GLN y Aplicaciones]{\Huge Generaci\'on de Lenguaje Natural y Aplicaciones}
%\mH{Lecture \#\mytitlenbr:} \mytitle}

\author[Areces \& Benotti]{
 Carlos Areces y Luciana Benotti\\[1ex]
\normalsize \url{{carlos.areces, luciana.benotti}@gmail.com}}

\institute[INRIA / UNC]{
INRIA Nancy Grand Est, Nancy, France\\
Universidad Nacional de C\'ordoba, C\'ordoba, Argentina}

\date{ELiC 2010 - Buenos Aires - Argentina}



\begin{document}

\begin{frame}
\frametitle{Bibliograf\'ia por Tema}

\mH{\bf Sistemas de GLN}
\medskip

\begin{thebibliography}{10}\small


\bibitem{Reiter94} Ehud Reiter.
\newblock Has a
  consensus NL generation architecture appeared, and is it
  psycholinguistically plausible?.
\newblock In \emph{Proceedings of INLG}, 1994.

\bibitem{ReiterDale00} Ehud Reiter and Robert Dale.
\newblock Building Natural Language Generation Systems.
\newblock Cambridge University Press, 2000.

\bibitem{Reiteretal05} Ehud Reiter, Somayajulu Sripada, Jim Hunter, Jin Yu, and Ian Davy
\newblock Choosing words in computer-generated weather forecasts.
\newblock In \emph{Artificial Intelligence}, 167:137-169, 2005.

\end{thebibliography}
\end{frame}


\begin{frame}
\frametitle{Bibliograf\'ia por Tema}

\mH{\bf Tree Adjoining Grammars}
\medskip

\begin{thebibliography}{10}\small

\bibitem{Joshi87} Aravind Joshi.
\newblock An introduction to tree adjoining
grammars.
\newblock In A. Manaster--Ramer (ed.) \emph{Mathematics of Language}, pp. 87--114. John Benjamins, 
Amsterdam, 1987.

\bibitem{ShankerJoshi85} Vijay Shanker and Aravind Joshi.
\newblock Some computational properties of tree adjoining grammars.
\newblock In \emph{Proceedings of ACL}, 1985.

\bibitem{Bresnanetal82} Joan  Bresnan, Ronald  Kaplan, Stanley Peters and
Annie Zaenen.
\newblock Cross--serial dependencies in Dutch.
\newblock \emph{Linguistic Inquiry}, 13:613--635, 1982.

\end{thebibliography}
\end{frame}

\begin{frame}
\frametitle{Bibliograf\'ia por Tema}

\mH{\bf Surface Realization}
\medskip

\begin{thebibliography}{10}\small

\bibitem{Carrolletal99} John Carroll, Ann Copestake, Dan Flickinger and Victor Poznanski.
\newblock An Efficient Chart Generator for (Semi-)Lexicalist Grammars.
\newblock In \emph{Proceedings of EWNLG}, 1999.

\bibitem{CarrollOepen05} John Carroll and Stephan Oepen.
\newblock High efficiency realization for a wide-coverage unification grammar.
\newblock In \emph{Proceedings of IJCNLP}, 2005.


\bibitem{GardentKow07} Claire Gardent and Eric Kow.
\newblock A Symbolic Approach to Near-Deterministic Surface Realisation using
Tree Adjoining Grammar.
\newblock In \emph{Proceedings of ACL}, 2007.

\end{thebibliography}
\end{frame}



\begin{frame}
\frametitle{Bibliograf\'ia por Tema}

\mH{\bf Expresiones Referenciales}
\medskip

\begin{thebibliography}{10}\small

\bibitem{Prince91} Ellen Prince.
\newblock Toward a taxonomy of given-new information.
\newblock In P. Cole, editor, Radical Pragmatics, pages 223--56. Academic Press, New York, 1981.

\bibitem{ClarkWilkes-Gibbs86} H. H. Clark and D. Wilkes-Gibbs. 
\newblock Referring as a collaborative process.
\newblock Cognition, 22:1--39, 1986.

\end{thebibliography}
\end{frame}


\begin{frame}
\frametitle{Bibliograf\'ia por Tema}
\mH{\bf Generaci\'on de Expresiones Referenciales}
\medskip

\begin{thebibliography}{10}\small

\bibitem{Dale89} Robert Dale.
\newblock Cooking up referring expressions.
\newblock In \emph{Proceedings of ACL}, 1989.

\bibitem{ReiterDale92} Ehud Reiter and Robert Dale.
\newblock A fast algorithm for the generation of referring expressions.
\newblock In \emph{Proceedings of Coling}, 1992.

\bibitem{DaleRaiter95} Robert Dale and Ehud Reiter.
\newblock Computational interpretations of the gricean maxims in the generation of referring expressions.
\newblock \emph{Cognitive Science}, 19(2):233--263, 1995.

\bibitem{StoneWebber98} Matthew Stone and Bonnie Webber.  
\newblock Textual economy through close coupling of syntax and semantics. 
\newblock In {Proceedings of INLG},  1998.

\end{thebibliography}
\end{frame}

\begin{frame}
\frametitle{Bibliograf\'ia por Tema}

\mH{\bf Generaci\'on de Expresiones Referenciales}
\medskip

\begin{thebibliography}{10}\small

\bibitem{Deemter02} Kees van Deemter.
\newblock Generating referring expres-
sions: Boolean extensions of the incremental algorithm.
\newblock In \emph{Computational Linguistics}, 28(1):37--52, 2002.

\bibitem{Krahmeretal03} Emiel Krahmer, Sebastiaan van Erk, and Andr\'e Verleg.
\newblock Graphbased generation of referring expressions.
\newblock In \emph{Computational Linguistics}, 29(1), 2003.

\bibitem{Arecesetal08} Carlos Areces, Alexander Koller and Kristina Striegnitz. 
\newblock Referring Expressions as Formulas of Description Logic. 
\newblock In \emph{Proceedings of INLG}, 2008.

\end{thebibliography}
\end{frame}


\begin{frame}
\frametitle{Bibliograf\'ia por Tema}

\mH{\bf Generaci\'on de Instrucciones en Entornos Virtuales}
\medskip

\begin{thebibliography}{10}\small

\bibitem{} Kees van Deemter and Jan Odijk.
\newblock Context modeling and the generation of spoken discourse.
\newblock \emph{Speech Communication}, 1997.

\bibitem{SluisKrahmer01} Ielka van der Sluis and Emiel Krahmer.
\newblock Generating referring expressions in a multimodal context.
\newblock In \emph{Proceedings of CLIN}, 2001.

\bibitem{ByronFosler-Lussier06} Donna Byron and Eric Fosler-Lussier.
\newblock The OSU Quake 2004 corpus of two-party situated problem-solving dialogs.
\newblock In \emph{Proceedings of LREC}, 2006.
  
\bibitem{Denis10} Alexandre Denis.
\newblock Generating Referring Expressions with Reference
Domain Theory.
\newblock In \emph{Proceddings of INLG}, 2010.

\end{thebibliography}
\end{frame}

\frametitle{Bibliograf\'ia por Tema}
\begin{frame}

\mH{\bf Evaluaci\'on}
\medskip

\begin{thebibliography}{10}\small

\bibitem{BelzGatt08} Anja Belz and Albert Gatt.
\newblock Intrinsic vs.\ Extrinsic Evaluation Measures for Referring Expression   Generation.
\newblock In \emph{Proceedings of ACL}, 2008.

\bibitem{ReiterBelz09} Ehud Reiter and Anja Belz.
\newblock An Investigation into the Validity of Some
Metrics for Automatically Evaluating Natural Language Generation
Systems.
\newblock In \emph{Computational Linguistics}, 25:529--558, 2009.
  
\bibitem{Kolleretal10} Alexander Koller, Kristina Striegnitz, Andrew Gargett, Donna Byron,
Justine Cassell, Robert Dale, Johanna Moore, and Jon Oberlander.
\newblock Report on the Second NLG Challenge on Generating Instructions in
Virtual Environments (GIVE-2).
\newblock In \emph{Proceedings of INLG}, 2010.

\bibitem{Garget10} A.\ Gargett, K.\ Garoufi, A.\ Koller, and K.\
Striegnitz.
\newblock The GIVE-2 Corpus of Giving Instructions in Virtual
Environments.
\newblock In \emph{Proceedings of LREC}, 2010.


\end{thebibliography}
\end{frame}

\end{document}
